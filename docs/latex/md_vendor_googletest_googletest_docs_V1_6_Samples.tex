If you\textquotesingle{}re like us, you\textquotesingle{}d like to look at some Google Test sample code. The \href{../samples}{\texttt{ samples folder}} has a number of well-\/commented samples showing how to use a variety of Google Test features.


\begin{DoxyItemize}
\item \href{../samples/sample1_unittest.cc}{\texttt{ Sample \#1}} shows the basic steps of using Google Test to test C++ functions.
\item \href{../samples/sample2_unittest.cc}{\texttt{ Sample \#2}} shows a more complex unit test for a class with multiple member functions.
\item \href{../samples/sample3_unittest.cc}{\texttt{ Sample \#3}} uses a test fixture.
\item \href{../samples/sample4_unittest.cc}{\texttt{ Sample \#4}} is another basic example of using Google Test.
\item \href{../samples/sample5_unittest.cc}{\texttt{ Sample \#5}} teaches how to reuse a test fixture in multiple test cases by deriving sub-\/fixtures from it.
\item \href{../samples/sample6_unittest.cc}{\texttt{ Sample \#6}} demonstrates type-\/parameterized tests.
\item \href{../samples/sample7_unittest.cc}{\texttt{ Sample \#7}} teaches the basics of value-\/parameterized tests.
\item \href{../samples/sample8_unittest.cc}{\texttt{ Sample \#8}} shows using {\ttfamily Combine()} in value-\/parameterized tests.
\item \href{../samples/sample9_unittest.cc}{\texttt{ Sample \#9}} shows use of the listener API to modify Google Test\textquotesingle{}s console output and the use of its reflection API to inspect test results.
\item \href{../samples/sample10_unittest.cc}{\texttt{ Sample \#10}} shows use of the listener API to implement a primitive memory leak checker. 
\end{DoxyItemize}